\documentclass[a4paper,11pt]{report}
\usepackage{graphics}
\usepackage{graphicx}
\usepackage{textcomp}
\usepackage{array,longtable}
\usepackage{listings}
\usepackage{float}
\usepackage{color}
\usepackage{amsmath}
\usepackage{xcolor}
\usepackage{blindtext}
\usepackage{pstricks}
\usepackage{pst-node}
\usepackage{pst-grad}
\usepackage{pst-coil}
\usepackage{pst-text}
\usepackage{minted}
\usepackage{etex}
\usepackage{array,longtable}
\usepackage{courier}

\lstset{basicstyle=\footnotesize\ttfamily,breaklines=true}
\lstset{framextopmargin=50pt}
%%%
% \usepackage{geometry}
% \geometry{
% a4paper,
% total={170mm,257mm},
% left=20mm,
% top=20mm,
% }
%%% 
 
% Title Page
\title{ConstraintGen Library for LLVM}
\author{Pushpinder Singh}

\begin{document}
\maketitle

\begin{abstract}

This document describes constraintgen library for generating points-to 
constraints for LLVM IR. Most of the points-to analysis depends on common 
properties of pointers present in the source code. These properties for each 
pointer in the source code can be gathered by using this library for LLVM IR.

\end{abstract}

\tableofcontents

\chapter{Introduction}

A number of points-to analysis have been developed. All these analysis have one 
thing in common that they depend upon some common properties of the pointers 
present in the source code of a program. These properties include information 
like the assignments to the pointers, type information, or level of indirection.
This information can represented in the form of set of constraints. These 
constraints correspond to the assignments in the source code. Only those 
assignments matter in which there is a write/read to pointer variable.

This library tries to extract this type of information in form of constraints 
from LLVM IR. It provides set of well defined APIs for generating constraints 
from LLVM IR.

\section{Motivation}
Most of the points-to analyses depend only on the pointer assignments in
the source code. Other assignments for such analyses doesn't matter because
they do not affect pointers. For implementing these analyses, the
first step would be to extract these assignments from the IR which is quite a 
waste of time if someone has to write a number of such analysis. So instead of
wasting time on writing code for extracting these statements, this library 
provides an API for doing so easily. The implementer now only has to focus on 
implementing the analysis.

This library has been designed by keeping in mind the forward and backward
analysis and top-down and bottom-up analyses. So it generates constraint
without depending on the flow or context information.

\chapter{Constraint Gen library}

Constraint gen library is designed to generate Points-to constraints from the
LLVM Intermediate Code. In order to generate the constraints, it classifies the
various LLVM IR instructions into four basic pointer assignments as described 
in Section 2.2. This library considers only those LLVM IR statements which are
relevant for \textit{Points-to Analysis} i.e. only those instructions in which 
there is 
read/write to a pointer. The library considers each pointer statement in the 
program independently. This each constraint is independent and is not 
influenced by control flow or call context. So, if someone want to do a 
forward flow analysis or backward flow analysis, or Context sensitive or 
insensitive, this library will be helpful for him.


\section{Design}
The design of this library is motivated by the fact that most of the points-to 
analysis depends on the pointers used in the source code. So it is designed 
such that it can cover the requirements of the most of the Points-to analyses.

All the information about a variable is stored in a Points-to Node 
(PointsToNode in implementation). Internally this library creates a mapping 
between those PointsToNode and LLVM's Value (every variable or instruction is 
represented by this Value). Each Points-to node is assigned a unique id. This 
id is an unsigned integer. Each constraint represents a single assignment in 
the IR. Section 2.3 discusses more about implementation level details 
of a single constraint.

This library tries to simplify the complex LLVM IR statements into four basic 
pointer assignments. Section 2.2 discusses about the classification of LLVM IR 
statements.

\section{Classification of LLVM Instructions}
Following four basic categories are used to classify the relevant LLVM 
Instructions:

\begin{itemize}
    \item Address Of Instructions, e.g. a = \&b
    \item Load Instructions, e.g.  a = *b
    \item Store Instructions, e.g. *a = b
    \item Copy Instructions, e.g. a = b
\end{itemize}

\subsection{Address Of Instructions}
Address of instructions are those instructions which refers to the assignment 
in which address of some variable is assigned to some other variable/pointer.
Address of instruction corresponds to following standard C statement,
    
\begin{lstlisting}
    a = &b
\end{lstlisting}

In LLVM, this is satisfied by alloca and getelementptr instruction.
Alloca instruction allocates memory in the stack and returns a pointer to it.
Whereas, getelementptr instruction takes a base address and computes the address
of element relative to base address using the provided arguments.
Example,

\begin{minted}{llvm}
%a = alloca i32, i32 4
%b = getelementptr i32, i32* %a, i32 1   ;select second element                 
 
                       
\end{minted}

\noindent
First statement is equivalent to, (here b is memory being allocated in stack)

\begin{lstlisting}
    a = &b
\end{lstlisting}

\noindent
And the second statement is equivalent to, (b should not be confused with 
previous example)
\begin{lstlisting}
    b = &a[1]
\end{lstlisting}

\noindent
Following statements are classified as address-of instructions
\begin{itemize}
    \item \texttt{alloca}
    \item \texttt{getelementptr}
\end{itemize}

\subsection{Load Instructions}
Load instructions are those assignments in which the address stored at the 
location pointed by a pointer is assigned to another pointer. So, we are 
basically dereferencing the pointer which points to another pointer and address 
stored in that pointer is assigned to other variable/pointer.
load instruction in LLVM can be used to represent following C statement,
\begin{lstlisting}
    a = *b
\end{lstlisting}

\noindent
Load instruction in LLVM takes an address and reads memory from that address,
e.g.,

\begin{minted}{llvm}
    %a = load i32*, i32** %b
\end{minted}

\noindent
Here \texttt{load} instruction will read the memory from the location pointed 
by \texttt{\%b} and will store the value read into the \texttt{\%a}.

\noindent
Instruction that satisfies this type is,
\begin{itemize}
 \item \texttt{load}
\end{itemize}


\subsection{Store Instructions}
Store instructions are those assignments in which address stored in a pointer 
is assigned to the location pointed by another pointer.
store instruction in LLVM is equivalent to following C statement,
\begin{lstlisting}
    *a = b
\end{lstlisting}

\noindent
Store instruction in LLVM writes to the memory at address provided as second
argument. First argument is the value which has to be written.
Example,
\begin{minted}{llvm}
    store i32* %b, i32** %a
\end{minted}

\noindent
Constraint-gen library ignores those store statements which do not write address
to memory or which involves writing non pointer values to the memory. So the
following store statements in the IR will be ignored,

\begin{lstlisting}
    store i32 10, i32* %a
\end{lstlisting}

\noindent
Instruction that satisfies this type is,
\begin{itemize}
 \item \texttt{store}
\end{itemize}


\subsection{Copy Instructions}
Copy instructions are those instructions which involve copying of the address 
from one SSA variable to another. For e.g. in phi instruction
\begin{lstlisting}
    %b = phi i32* [ %a, %one ], [ %c, %two ]
\end{lstlisting}

\noindent
It is equivalent to the following C statement,
\begin{lstlisting}
    a = b;
\end{lstlisting}

\noindent
Following instructions satisifies this constraint,
\begin{itemize}
    \item \texttt{phi}
    \item \texttt{select}
    \item \texttt{extractvalue}
    \item \texttt{insertvalue}
\end{itemize}

More details about the constraint being generated for each instruction are
mentioned in next section.

\section{Constraints and their Generation}
Constraints are the information about the pointers assignemnts used in a 
particular IR statement. A single constraint refers to a single possible 
assignment. This assignment can be one of above four type. Each constraint has 
the following properties,

\begin{itemize}
    \item LHS node id
    \item RHS node id
    \item LLVM statement
    \item Type of the constraint, i.e. address of, or load, or store, or copy
\end{itemize}

\noindent
Both LHS and RHS node consists of atleast the following information. 
(Implementation details are discussed in Section 2.4).

\begin{itemize}
    \item the corresponding \texttt{llvm::Value}
    \item what other nodes are being used (See section ??)
    \item Type information, whether it is a struct or an array or a normal 
pointer
    \item A unique identifier
\end{itemize}

\subsection{Generation of constraints}
For each statement type, respective method is used to extract constraint from
that statement type. Constraints are generated for only following instructions.
Other instructions are irrevelant since they do not use pointers or involve
assignments that may affect pointers.

\begin{itemize}
    \item \texttt{alloca} instruction
    \item \texttt{load} instruction
    \item \texttt{store} instruction
    \item \texttt{getelementptr} instruction
    \item \texttt{phi} instruction
    \item \texttt{bitcast} instruction
    \item \texttt{select} instruction
    \item \texttt{extractvalue} instruction
    \item \texttt{insertvalue} instruction
\end{itemize}

\subsubsection{\texttt{alloca} instruction}
Alloca instruction falls under the category of address-of instruction. Alloca 
instruction allocates the memory in stack taking the type and number of 
elements as arguments. It returns the pointer to that memory. 
Example,

\begin{lstlisting}
    %a = alloca i32
\end{lstlisting}
\noindent
Above instruction allocates the memory for i32 value and returns a pointer to 
it. \%a is of type i32*. Constraint that is generated will be of the following 
form,

\begin{lstlisting}
    { lhs-id } = & { rhs-id }
\end{lstlisting}
\noindent
where \texttt{lhs-id} and \texttt{rhs-id} are the id's of the variables. 
\texttt{lhs-id} represents the \texttt{\%a} and \texttt{rhs-id} represents the 
node whose address is being assigned. In LLVM, this is a virtual node. So, 
library creates a dummy node.
Example,

\begin{lstlisting}
    1 = & 0
\end{lstlisting}


\subsubsection{\texttt{load} instruction}
Only those load instructions whose return type is a pointer type are used for
generating constraint. For e.g for the following load instruction

\begin{lstlisting}
    %a = load i32*, i32** %b
\end{lstlisting}

\noindent
RHS node is \%b and LHS node is \%a. And the type of instruction is load.

\subsubsection{\texttt{store} instruction}
In case of store instruction, only those instructions are considered in which
there is write to a pointer value. For e.g.

\begin{lstlisting}
    store i32* %a, i32** %b
\end{lstlisting}
\noindent
and the following store instruction will not be considered for constraint

\begin{lstlisting}
    store i32 10, i32* %a
\end{lstlisting}

In the first example, RHS node is \%a and LHS node is \%b. And the type of 
instruction is store.

\subsubsection{\texttt{getelementptr} instruction}
getelementptr instruction in LLVM is used for calculating the address 
of a member elements of a aggregate type. So it takes the base address of
the element of an aggregate type (e.g. struct or array) as the first argument.
Indices are provided in the form of comma separated list following the base 
address.
e.g.

\begin{lstlisting}
    %a = getelementptr { i32*, i32 }, { i32*, i32 }* %b, i32 0, i32 0
\end{lstlisting}

getelementptr instruction generates special kind of constraint in which LHS is 
same as \%a but RHS is a special node consisting of list of nodes being used as 
arguments. So, LHS here is \%a, and RHS is a node consisting of \%b, `i32 0' 
and 
`i32 0' in its use list. Or we can say that RHS uses \%b, i32 0 and i32 0. And 
the type of instruction is copy.

\subsubsection{\texttt{phi} instruction}
Phi instruction maintains the list of all the incoming values in its use list
as its RHS. While LHS is same as in LLVM statement.
E.g.
\begin{lstlisting}
    %a = phi i32* [ %0, %one ], [ %1, %two ], [ %3, %three ]
\end{lstlisting}

Phi instruction generates multiple constraints. Each of these constraint 
consists of single possible assignment to the LHS. So the constraints generated 
for above instruction will be, \%a = \%0, \%a = \%1, \%a = \%3. Each of these 
constraints is a copy assignment.

\subsubsection{\texttt{bitcast} instruction}
bitcast instruction can convert a pointer of one type to another type. It is
essentially a copy operation. LHS be the new pointer and RHS will be the old 
pointer.

E.g.
\begin{lstlisting}
    %a = bitcast i32* %b to i64*
\end{lstlisting}

\subsubsection{\texttt{select} instruction}
select instruction is similar to br instruction. So, instead of jumping, it
selects a value from a pair based on a condition variable. It also generates 
multiple constraints where each constraint is a possible assignment. Each 
constraint represents a copy assignment.

E.g.
\begin{lstlisting}
    %c = select i1 %cond, i32* %a, i32* %b
\end{lstlisting}

Constraints generated for this instruction will be, \%c = \%a \& \%c = \%b

\subsubsection{\texttt{extractvalue} instruction}
ExtractValue instruction is used to extract the value of a member element from 
an object of an aggregate type (struct or array). Whereas getelementptr is used 
to calculate the address of a member element with respect to the base address, 
the extractvalue instruction is used to directly access the value having a 
member element. It takes the object directly as the first argument and next 
arguments are the list of indices representing the index of the member 
elements of an aggregate type.

For e.g. if \%a is of type { i32*, i32 }, then 
following instruction will extract the address being holded by first member 
which is a pointer.

\begin{lstlisting}
 %first = extractvalue { i32*, i32 } %a, 0
\end{lstlisting}

Constraint generated is similar to the that of getelementptr instruction but 
instead it is of type copy. RHS contains the \%a node and in its use list, it 
contains the indices directly. So, unlike getelementptr, it doesn't store the 
id's of the node being used because indices are constants here. But in 
getelementptr instruction, indices can be variable too.

\subsubsection{\texttt{insertvalue} instruction}
Insertvalue instruction is used to insert a value into the member element of an 
object of an aggregate type.
The first operand of an ‘insertvalue‘ instruction is a value of struct or array 
type. The second operand is a first-class value to insert. The following 
operands are constant indices indicating the position at which to insert the 
value in a similar manner as indices in a ‘extractvalue‘ instruction. The value 
to insert must have the same type as the value identified by the indices.

For e.g.,

\begin{lstlisting}
 %aggregate = insertvalue { i32*, i32 } %a, i32* %b, 0
\end{lstlisting}

The above code is similar to the following C code,

\begin{lstlisting}
 t = a;
 t.first = b;
 aggregate = t;
\end{lstlisting}

The constraint generated is of type copy as it copies the address stored in \%b 
to a member of an object of aggregate type. RHS node of the constraint will be 
same as that of insertvalue instruction.

\section{Instructions not handled}
Following instructions are not being handled by the library,

\begin{itemize}
    \item call
    \item ret
    \item inttoptr
    \item ptrtoint
\end{itemize}

\chapter{Implementation Details}
In this chapter, implementation level details of this are provided. Various 
data structures used and APIs provided are explained in the next sections.

\section{Data Structures}
In this section, various classes and their properties used in the library are 
explained.

\subsection{\texttt{PointsToNode}}
This is an important class which holds the information regarding single 
pointer/variable. Though as the name suggests that this class represents 
pointers only, but in some cases it can represent non-pointers as well.
Every object of this class is assigned a unique id. This id is used in place 
of using object directly. Thus helps in easier debugging.

\subsection{\texttt{NodeIndex}}
This type is used for ID's. Every \texttt{PointsToNode} object is assigned
a unique ID of type \texttt{NodeIndex} which is equivalent to \texttt{unsigned int}.

\subsection{\texttt{Constraint}}
This class holds the information of single possible assignment in an 
instruction. It holds the id's of two PointsToNode objects and the type of the 
assignment which can be one of the four possible assignments as described 
in Section 2.2.

\subsection{\texttt{ConstraintBuilder}}
This class provides APIs for generating constraints from LLVM IR. This class 
also provides public methods to generates ID's for LLVM pointers.

\subsection{\texttt{PointerSymbolTable}}
This class manages all the tables and various mappings between library's 
objects and LLVM's objects. This class is used by ConstraintBuilder in order to 
generate ID's.

\paragraph{}
More details about each class' members and public methods are described in 
documentation generated by doxygen from source directly.

\section{Public API's provided}
\texttt{ConstraintBuilder} class provides interface for generating constraints
and generating ID's for LLVM IR. These methods are described below.

\subsection{\texttt{getConstraint}}
\paragraph{Prototype}
\begin{minted}{C++}
std::vector<Constraint>
    getConstraint(llvm::Instruction *i);
\end{minted}

\paragraph{Description:}
This function takes pointer to the LLVM Instruction object and generates a list
of constraints. For most of the instructions, except some copy instructions,
this only generates single constraint. It returns the list of constraints. If
the list is empty then it means that instructions was invalid for generating
constraint.

\subsection{\texttt{generateId}}
Since the library does not handles \texttt{call/invoke} instructions so it provides
this function for assigning ID's to the LLVM variables which were missed in the call
instruction by the library.

\paragraph{Prototype}
\begin{minted}{C++}
NodeIndex generateId(llvm::Value *val);
\end{minted}

\paragraph{Description:}
This function takes a pointer to LLVM Value object and creates a \texttt{PointsToNode}
object and returns the ID of that object.

\subsection{\texttt{makeConstraint}}
\paragraph{Prototype}
\begin{minted}{C++}
Constraint makeConstraint(NodeIndex lhs,
                          NodeIndex rhs,
                          ConstraintType type);
\end{minted}

\paragraph{Description}
This function takes three arguments. First two are the ID's of the lhs and rhs
respectively and the last argument is the \texttt{ConstraintType} which can be
one of the following:
\begin{itemize}
    \item \texttt{kInvalid}
    \item \texttt{kAddressOf}
    \item \texttt{kLoad}
    \item \texttt{kStore}
    \item \texttt{kCopy}
\end{itemize}

\end{document}
