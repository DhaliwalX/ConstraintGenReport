\documentclass{beamer}

\usepackage[utf8]{inputenc}
\usepackage{graphicx}
\usepackage{booktabs}
\usepackage[english]{babel}
\usepackage{pstricks}
\usepackage{etex}
\usepackage{cancel}
\usepackage{array}
\usepackage{pst-node}
\usepackage{pst-grad}
\usepackage{pst-text}
\usepackage{pst-coil}
\usepackage{pst-rel-points}
\usepackage{colortbl}
\usepackage{comment}
\usepackage{amssymb}
\usepackage{amsmath}
\usepackage{stmaryrd}
\usepackage{amssymb}
\usepackage{amsmath}
\usepackage{xspace}
\usepackage{listings}
\usepackage{multirow}
\usepackage{multicol}
\usepackage{graphicx}
\usepackage[normalem]{ulem}
\usepackage{array,longtable}

\usepackage{verbatim}
\usepackage{tikz}
\usepackage{fmtcount}
\usepackage{scalefnt}
\usepackage{wasysym}
\usepackage{stmaryrd}
\usepackage{pgf,pgfarrows,pgfnodes,pgfautomata,pgfheaps}
\usepackage{pgfplotstable}
\usepackage{pgfplots}
\usepackage{isomath}
\usetikzlibrary{patterns}

\title{Constraint Generation Library for LLVM}
\author[Pushpinder Singh]{Pushpinder Singh\\~}
\institute[IIT Bombay]
{
\textit{Under the guidance of}\\
\textbf{Prof. Uday Khedker}\\
Computer Science Department,\\
IIT Bombay
}

\date{2017}

\begin{document}

\frame{\titlepage}

\begin{frame}
\frametitle{Contents:}
\tableofcontents
\end{frame}

\section{Introduction}

% about LLVM
\begin{frame}
\frametitle{A bit about LLVM}

\begin{itemize}
    % LLVM started as a research project in University of Illinois by Chris Lattner
    % He designed LLVM as a framework for making program analysis easier. Slowly
    % the project developed into full compiler framework. Now this project has
    % full C/C++ compiler known as clang which is used as default compiler in
    % Apple MacOS and iOS and in many unix operating systems.
    \item<1-> Framework for building compilers
    % AS previously told the framework is designed for making program analysis easier
    \item<2-> Framework for Program Analysis
    % LLVM is designed very well. This project is divided into many independent sub projects.
    % Thus integrating LLVM into another project is much easier.
    \item<3-> Provides reusable components
    % LLVM is designed for making program analysis easier. It provides a huge set
    % of compiler optimizations. These optimizations can be compile-time or link-time
    % or run-time
    \item<4-> Designed for compile-time, link-time, run-time optimizations
    % LLVM is designed as language independent framework. It provides a well designed
    % intermediate representation that satisfies the requirement of most of the programming
    % languages. We will discuss about IR in next slide
    \item<5-> Language independent design
\end{itemize}
\end{frame}

% about llvm IR
\begin{frame}
\frametitle{A bit about LLVM}
% in this presentation we will discuss about LLVM IR.
Overview of LLVM IR,
\begin{itemize}
    % LLVM IR is Static Single Assignment based representation. A variable
    % is assigned only once and can not be changed 
    \item<1-> SSA based representation
    % IR is completely type safe.
    \item<2-> Type safe
    % The LLVM IR is near assembly representation but is easier than assembly.
    % It has infinite set of virtual registers. Every variable is a virtual register.
    \item<3-> Low-level operations
    % LLVM IR is capable of representing all the high-level languages like C, C++
    % objective-C, swift, javascript, java etc
    \item<4-> Capable of representing all the high-level languages
\end{itemize}

\end{frame}

\section{Points-to Analysis}

\begin{frame}
\frametitle{Points-to Analysis}
- Points to analysis is an analysis which involves determining the association between
pointer variables and the locations pointed to by them.

\vspace{15pt}

\pause
- Points-to Analyses works on pointer assignments in a source program.

\pause
Examples,
\begin{itemize}
    \item<1-> \texttt{a = \&b}
    \item<2-> \texttt{a = *b}
    \item<3-> \texttt{a[10] = f.x[0].g}
\end{itemize}

\vspace{15pt}

\pause
\alert{- We need to extract those assignments.}

\end{frame}

\begin{frame}
\frametitle{Points-to Analysis}
Those pointer assignments are termed as Constraints.
\end{frame}

\begin{frame}
\frametitle{Constraints}
\vspace{15pt}
\huge What are constraints?

\vspace{15pt}
\normalsize
Constraint is single basic pointer assignment in a source program.
Constraint has the following information:
\begin{itemize}
    \item<2-> LHS pointer
    \item<3-> RHS pointer
    \item<4-> and type of assignments
\end{itemize}
\end{frame}

\begin{frame}
\frametitle{Types of constraints}

\begin{itemize}
    \item<1-> Address of. For e.g a = \&b
    \item<2-> Copy. For e.g. a = b
    \item<3-> Load. For e.g. a = *b
    \item<4-> Store. For e.g. *a = b
\end{itemize}

\end{frame}

\frame{
\frametitle{Instructions in LLVM IR for Points-to Analysis}
\psset{unit=1mm}
\begin{pspicture}(0,0)(120,80)
 \onslide<1->
 {
 \putnode{n1}{origin}{20}{75}{%
         {\blue Instructions in LLVM IR}}
  }
  \onslide<2->
{
 \putnode{n2}{n1}{0}{-8}{%
         {alloca}}
 \putnode{n3}{n2}{0}{-8}{%
         {getelementptr}}
 \putnode{n4}{n3}{0}{-8}{%
         {load}}
 \putnode{n5}{n4}{0}{-8}{%
         {store}}
 \putnode{n6}{n5}{0}{-8}{%
         {phi}}
 \putnode{n7}{n6}{0}{-8}{%
         {select}}
 \putnode{n8}{n7}{0}{-8}{%
         {extractvalue}}
 \putnode{n9}{n8}{0}{-8}{%
         {insertvalue}}
}   

\onslide<3->
 {
 \putnode{n10}{origin}{80}{75}{%
         {\blue Primitive Statements for Points-to Analysis}}
  }
  \onslide<4->
{
 \putnode{n11}{n10}{0}{-15}{%
         {\red Address of: } \textit{x = \&y}}
 \putnode{n12}{n11}{0}{-15}{%
         {\red Load: }\textit{x = *y}}
 \putnode{n13}{n12}{0}{-15}{%
         {\red Store: }\textit{*x = y}}
 \putnode{n14}{n13}{0}{-15}{%
         {\red Copy: }\textit{x = y}}
} 
\onslide<5->
{
\ncline[linestyle=dashed,dash=.4 .4,nodesep=1]{->}{n2}{n11}
\ncline[linestyle=dashed,dash=.4 .4,nodesep=1]{->}{n3}{n11}
\ncline[linestyle=dashed,dash=.4 .4,nodesep=1]{->}{n4}{n12}
\ncline[linestyle=dashed,dash=.4 .4,nodesep=1]{->}{n5}{n13}
\ncline[linestyle=dashed,dash=.4 .4,nodesep=1]{->}{n6}{n14}
\ncline[linestyle=dashed,dash=.4 .4,nodesep=1]{->}{n7}{n14}
\ncline[linestyle=dashed,dash=.4 .4,nodesep=1]{->}{n8}{n14}
\ncline[linestyle=dashed,dash=.4 .4,nodesep=1]{->}{n9}{n14}

}
\end{pspicture}
}


\begin{frame}
\frametitle{ConstraintGen library}
% so when a developer starts writing a points-to analysis, his first
% step is to extract those assignments and put them in a suitable
% data structure. This involves handling a lot of cases.
% This library tries to handle all the constraint extraction part.
% Hence it is easier for the developer to focus on the main analysis.
ConstraintGen library extracts those assignments from the LLVM IR and puts
them in a well designed data structure.

\vspace{15pt}
\pause
It handles all types of assignments present in the LLVM IR.

\end{frame}

\begin{frame}
\frametitle{ConstraintGen library continued...}

Information about every pointer is stored in \textbf{PointsToNode} data structure.
% This data structure stores information like Type information, corresponding LLVM value
% from which this node was created.

\vspace{15pt}
\pause
Every node is assigned a unique ID. % which is an unsigned int

\vspace{15pt}
\pause
Multiple constraints are possible from single LLVM instruction.
\begin{itemize}
    \item select instruction
    \item extractvalue instruction
    \item insertvalue instruction
    \item phi instruction
\end{itemize}

\vspace{15pt}
\pause
Context and Flow Insensitive
% Hence constraints can be generated independently.

\end{frame}

\begin{frame}
\frametitle{Constraints generated by the Library}
Constraints generated by library have following properties,
\begin{itemize}
    \item<1-> ID of the LHS Node
    \item<2-> ID of the RHS Node
    \item<3-> Type of the constraint
\end{itemize}

\end{frame}


\begin{frame}[fragile]
\frametitle{Examples}

\begin{columns}

\begin{column}{0.5\textwidth}
\begin{lstlisting}
define void @f() {
    %a = alloca i32
    ret void
}
\end{lstlisting}
\end{column}

\begin{column}{0.5\textwidth}
\begin{lstlisting}
    1 = & 0
\end{lstlisting}

\end{column}

\end{columns}
\end{frame}

\begin{frame}[fragile]
\frametitle{Examples}

\begin{columns}

\begin{column}{0.5\textwidth}
\vspace{12pt}
\begin{lstlisting}
define void @f() {
    %a = alloca i32
    %b = alloca i32*
    store i32* %a, i32** %b
    ret void
}
\end{lstlisting}
\end{column}

\begin{column}{0.5\textwidth}
\begin{lstlisting}
    1 = & 0
    3 = & 2
    * 3 = 1
\end{lstlisting}

\end{column}

\end{columns}
\end{frame}


\begin{frame}[fragile]
\frametitle{Examples}

define void @foo \(\) \{ \\
\hspace{15pt} \green \%a = alloca \{ i32*, i32 \} \\
\hspace{15pt} \normalcolor   \%b = alloca i32 \\
\hspace{15pt}    \%1 = getelementptr \{ i32*, i32 \}, \\
\hspace{30pt}            \{i32*, i32\}* \%a, i32 0, i32 0 \\
\hspace{15pt}    store i32* \%b, i32** \%1 \\
\hspace{15pt}    ret void \\
\}

\vspace{15pt}
Constraints:\\
\green 1 = \& 0 \\
\normalcolor    3 = \& 2 \\
    6 = \& 4 uses: \{ 1, 5, 5 \} \\
    * 6 = 3 \\
\end{frame}


\begin{frame}[fragile]
\frametitle{Examples}

define void @foo \(\) \{ \\
\hspace{15pt}  \%a = alloca \{ i32*, i32 \} \\
\hspace{15pt} \green   \%b = alloca i32 \\
\hspace{15pt} \normalcolor   \%1 = getelementptr \{ i32*, i32 \}, \\
\hspace{30pt}            \{i32*, i32\}* \%a, i32 0, i32 0 \\
\hspace{15pt}    store i32* \%b, i32** \%1 \\
\hspace{15pt}    ret void \\
\}

\vspace{15pt}
Constraints:\\
 1 = \& 0 \\
\green    3 = \& 2 \\
\normalcolor    6 = \& 4 uses: \{ 1, 5, 5 \} \\
    * 6 = 3 \\
\end{frame}


\begin{frame}[fragile]
\frametitle{Examples}

define void @foo \(\) \{ \\
\hspace{15pt}  \%a = alloca \{ i32*, i32 \} \\
\hspace{15pt}    \%b = alloca i32 \\
\hspace{15pt} \green   \%1 = getelementptr \{ i32*, i32 \}, \\
\hspace{30pt}      \{i32*, i32\}* \%a, i32 0, i32 0 \\
\hspace{15pt}  \normalcolor  store i32* \%b, i32** \%1 \\
\hspace{15pt}    ret void \\
\}

\vspace{15pt}
Constraints:\\
 1 = \& 0 \\
    3 = \& 2 \\
\green    6 = \& 4 uses: \{ 1, 5, 5 \} \\
\normalcolor    * 6 = 3 \\
\end{frame}


\begin{frame}[fragile]
\frametitle{Examples}

define void @foo \(\) \{ \\
\hspace{15pt}  \%a = alloca \{ i32*, i32 \} \\
\hspace{15pt}    \%b = alloca i32 \\
\hspace{15pt}    \%1 = getelementptr \{ i32*, i32 \}, \\
\hspace{30pt}      \{i32*, i32\}* \%a, i32 0, i32 0 \\
\hspace{15pt} \green   store i32* \%b, i32** \%1 \\
\hspace{15pt}  \normalcolor  ret void \\
\}

\vspace{15pt}
Constraints:\\
 1 = \& 0 \\
    3 = \& 2 \\
    6 = \& 4 uses: \{ 1, 5, 5 \} \\
\green * 6 = 3 \\
\end{frame}

\begin{frame}
\frametitle{Integrating the library}

\begin{itemize}
    \item<1-> Inside the tests directory of the ConstraintGen library
    \item<2-> Using the library inside the Analysis code
\end{itemize}

\end{frame}

\begin{frame}
\frametitle{Conclusion}
\begin{itemize}
    \item<1-> Covers requirements of the most of the Points-to Analysis
    \item<2-> Has been used in implementation of GPG Analysis
\end{itemize}
\end{frame}

\begin{frame}

\huge Thank you...
\end{frame}

\end{document}
