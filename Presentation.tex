\documentclass{beamer}

\usepackage[utf8]{inputenc}
\usepackage{graphicx}
\usepackage{booktabs}
\usepackage[english]{babel}

\title{Constraint Generation Library for LLVM}
\author[Pushpinder Singh]{Pushpinder Singh\\~}
\institute[IIT Bombay]
{
\textit{Under the guidance of}\\
\textbf{Prof. Uday Khedker}\\
Computer Science Department,\\
IIT Bombay
}

\date{2017}

\begin{document}

\frame{\titlepage}

\begin{frame}
\frametitle{Contents:}
\tableofcontents
\end{frame}

\section{Introduction}

% about LLVM
\begin{frame}
\frametitle{A bit about LLVM}

LLVM stands for Low-Level Virtual Machine.
\begin{itemize}
	% LLVM started as a research project in University of Illinois by Chris Lattner
	% He designed LLVM as a framework for making program analysis easier. Slowly
	% the project developed into full compiler framework. Now this project has
	% full C/C++ compiler known as clang which is used as default compiler in
	% Apple MacOS and iOS and in many unix operating systems.
	\item<1-> Framework for building compilers
	% AS previously told the framework is designed for making program analysis easier
	\item<2-> Framework for Program Analysis
	% LLVM is designed very well. This project is divided into many independent sub projects.
	% Thus integrating LLVM into another project is much easier.
	\item<3-> Provides reusable components
	% LLVM is designed for making program analysis easier. It provides a huge set
	% of compiler optimizations. These optimizations can be compile-time or link-time
	% or run-time
	\item<4-> Designed for compile-time, link-time, run-time optimizations
	% LLVM is designed as language independent framework. It provides a well designed
	% intermediate representation that satisfies the requirement of most of the programming
	% languages. We will discuss about IR in next slide
	\item<5-> Language independent design
\end{itemize}
\end{frame}

% about llvm IR
\begin{frame}
\frametitle{A bit about LLVM}
% in this presentation we will discuss about LLVM IR.
Overview of LLVM IR,
\begin{itemize}
	% LLVM IR is Static Single Assignment based representation. A variable
	% is assigned only once and can not be changed 
	\item<1-> SSA based representation
	% IR is completely type safe.
	\item<2-> Type safe
	% The LLVM IR is near assembly representation but is easier than assembly.
	% It has infinite set of virtual registers. Every variable is a virtual register.
	\item<3-> Low-level operations
	% LLVM IR is capable of representing all the high-level languages like C, C++
	% objective-C, swift, javascript, java etc
	\item<4-> Capable of representing all the high-level languages
\end{itemize}

\end{frame}

\section{Points-to Analysis}

\begin{frame}
\frametitle{Points-to Analysis}
- Points to analysis is an analysis which involves determining the association between
pointer variables and the locations pointed to by them.

\vspace{15pt}

\pause
- Points-to Analyses works on pointer assignments in a source program.

\pause
Examples,
\begin{itemize}
	\item<1-> \texttt{a = \&b}
	\item<2-> \texttt{a = *b}
	\item<3-> \texttt{a[10] = f.x[0].g}
\end{itemize}

\vspace{15pt}

\pause
\alert{- We need to extract those assignments.}

\end{frame}

\begin{frame}
\frametitle{Points-to Analysis}
Those pointer assignments are termed as Constraints.
\end{frame}

\begin{frame}
\frametitle{Constraints}
\vspace{15pt}
\huge What are constraints?

\vspace{15pt}
\normalsize
Constraint is single basic pointer assignment in a source program.
Constraint has the following information:
\begin{itemize}
	\item<2-> LHS pointer
	\item<3-> RHS pointer
	\item<4-> and type of assignments
\end{itemize}
\end{frame}

\begin{frame}
\frametitle{ConstraintGen library}
% so when a developer starts writing a points-to analysis, his first
% step is to extract those assignments and put them in a suitable
% data structure. This involves handling a lot of cases.
% This library tries to handle all the constraint extraction part.
% Hence it is easier for the developer to focus on the main analysis.
ConstraintGen library extracts those assignments from the LLVM IR and puts
them in a well designed data structure.

\vspace{15pt}
\pause
It handles all types of assignments present in the LLVM IR.

\end{frame}

\section{Using the library}
\begin{frame}
\frametitle{Using the library}
\end{frame}

\end{document}
